\documentclass{llncs}       % onecolumn (second format)

\usepackage{listings}
\usepackage[longend,ruled]{algorithm}
\usepackage{algorithmic}
\usepackage{fancyvrb}
\usepackage{graphicx}
\usepackage{tikz}
%\usepackage{refs,xspace}
%\usepackage{macros}
\usepackage{times}
%\usepackage{color,graphicx,enumerate}
%\usepackage{amsmath,amssymb,amsfonts,xspace}
\usepackage{url}
%\usepackage[T1]{fontenc}
\usepackage{floatflt}
\usepackage{epsfig,pifont}

\usepackage[square,sort&compress]{natbib}
\usepackage{mdframed}
\usepackage{subcaption}
\usepackage{xcolor}
\usepackage{xspace}
\usepackage{amssymb,latexsym,amsfonts,amsmath}
\usepackage{wrapfig}
\usepackage{hyperref}
\usepackage{paralist} 
\usepackage{booktabs}
\usepackage{multirow}
%\usepackage{smallsec}
%\usepackage[pdftex]{graphicx}
%\DeclareGraphicsExtensions{.pdf}
%\renewcommand{\baselinestretch}{0.95} 
\usepackage[ruled,shortend,algo2e]{algorithm2e}
\usepackage{comment}

\definecolor{mygray}{rgb}{0.95,0.95,0.95}
\definecolor{darkgreen}{rgb}{0,.45,0}
\lstset{language=C}
\lstset{backgroundcolor=\color{mygray}}
\lstset{basicstyle=\ttfamily\small}
\lstset{numbers=left, numberstyle=\small, stepnumber=1, numbersep=10pt}
\lstset{keywordstyle=\color{red}\emph} %\bfseries
\lstset{commentstyle=\color{darkgreen}}
\lstset{numberbychapter=false}
% these are the syntax-highlighting color definitions we will use.
\lstdefinestyle{customc}{
  belowcaptionskip=1\baselineskip,
  breaklines=true,
  frame=single,
  language=C,
  captionpos=b,
  otherkeywords={uint8_t, assert, assume, bool, havoc, sensitive},
  showstringspaces=false,
  numbers=left,
  basicstyle=\ttfamily,
  keywordstyle=\bfseries\color{green!40!black},
  commentstyle=\color{purple!40!black},
  identifierstyle=\color{blue},
  stringstyle=\color{orange},
}
\lstset{escapechar=@,style=customc}

\newcommand{\code}[1]{$\mathtt{#1}$}
\newcommand{\mcode}[1]{\mathtt{#1}}

\usetikzlibrary{arrows,backgrounds,decorations,decorations.pathmorphing,positioning,fit,automata,shapes.multipart,shapes,snakes}
\usetikzlibrary{patterns,matrix}
\usepackage[normalem]{ulem}
%\newenvironment{Example}{\begin{example}}{\hfill$\blacksquare$\end{example}}
\setlength{\abovecaptionskip}{2pt}
\addtolength{\textfloatsep}{-1ex}
%\usepackage{flushend}
% To eliminate orphans & widows
\clubpenalty = 10000
\widowpenalty = 10000
\displaywidowpenalty = 10000

\begin{document}

\newcommand{\limp}{\Rightarrow}
\newcommand{\lpmi}{\leftarrow}
\newcommand{\liff}{\leftrightarrow}

\newcommand{\Land}{\bigwedge}
\newcommand{\Lor}{\bigvee}

\newcommand{\unsat}{\textsc{unsat}}
\newcommand{\sat}{\textsc{sat}}

\newcommand{\true}{\mathit{true}}
\newcommand{\false}{\mathit{false}}
\newcommand{\Tr}{\mathit{Tr}}
\newcommand{\Bad}{\mathit{Bad}}
\newcommand{\Init}{\mathit{Init}}
\newcommand{\Inv}{\mathit{Inv}}

\newcommand{\Ifc}{\textsc{Ifc}\xspace}
\newcommand{\Bmc}{\textsc{Ifc-BMC}\xspace}

\newcommand{\yv}[1]{[{\bf YV:} #1]}

\newcommand{\vF}{\vec{F}}
\newcommand{\vG}{\vec{G}}
\newcommand{\vH}{\vec{H}}

\newcommand{\cT}{\mathcal{T}}
\title{Lazy Self-Composition for Security Verification}

%\author{Blank for double-blind review} 
\author{ Weikun Yang \and 
         Pramod Subramanyan \and 
         Yakir Vizel \and
         Aarti Gupta \and 
         Sharad Malik} 

\institute{Princeton University}

\maketitle

\begin{abstract}
\begin{comment}
We present a novel path-sensitive information flow analysis technique.

Secure information flow analysis is meant to detect if high-security assets can leak during an execution of a program. One approach for performing a precise information flow analysis, is by a reduction to safety verification by means of \emph{self-composition}. However, this reduction is intractable since it requires two copies of the program to be created, on which a safety property is checked. A second, less precise approach, is by using \emph{taint analysis}. Taint analysis provides a sound approximation for secure information flow analysis. While more efficient than self-composition, this approach often results in many ``false'' alarms, i.e. reporting leaks that do not exist.

In order to bridge the gap between the above two approaches, in this paper we present \Ifc, a novel, precise and efficient, secure information flow analysis technique. \Ifc relies on an interplay between a symbolic path-sensitive taint analysis and self-composition, where taint analysis guides self-composition and vice-versa. As a result of this interplay, full self-composition is never required, yet, the overall technique is sound and complete.

We implemented a prototype of \Ifc on top of SeaHorn and evaluated it on challenging examples. The experimental results show the potential of our approach.


%% aarti's suggestion below

We present a novel algorithm for verifying secure information flow that guarantees the absence of information leaks from high-security inputs to low-security outputs in a program. A classic approach for verifying secure information flow is to reduce the problem to safety verification on a \emph{self-composition}, where two copies of the program are created~\cite{BartheCsfw04}.
Although this reduction is sound and complete, it is challenging to perform safety verification in practice, especially on two copies of a program.
Another, less precise approach, is by using \emph{taint analysis} based on an information flow type system. Taint analysis provides a sound approximation for secure information flow. While more scalable than the self-composition approach, it often results in many false alarms, i.e., reporting leaks that do not exist.

To bridge the gap between these two approaches, in this paper we present \Ifc, a novel, precise, and efficient verifier for secure information flow. \Ifc relies on an interplay between a symbolic path-sensitive taint analysis and self-composition, where taint analysis guides self-composition and vice-versa. As a result of this interplay, full self-composition is never required, yet, the overall technique is sound and complete.
We implemented a prototype of \Ifc on top of SeaHorn~\cite{Seahorn} and evaluated it on challenging examples. The experimental results show the potential of our approach.
\end{comment}

%%%%%%%%%%%%%

The secure information flow problem has many applications in checking security properties in programs, e.g., that there are no information leaks from high-security inputs to low-security outputs. 
In this paper, we present \Ifc, a novel, precise, and efficient verifier for secure information flow. \Ifc improves the classical  self-composition approach (due to Barthe et al.) by making it lazier, in contrast to an eager upfront translation into two copies of a program. This lazy duplication is guided by taint analysis.
We implemented a prototype of \Ifc on top of SeaHorn tool and evaluated it on challenging examples. The experimental results show the potential of our approach.

\end{abstract}

\section{Introduction}
\label{sec:intro}
% intro.tex

{\bf AG:} have added basic points to highlight, terminology needs to be cleaned up, will add more background and highlights after experiments \\

One approach for verifying secure information flow is to convert it into a safety verification problem on a ``self-composition'' of the program, where two copies of the program are created on which a safety property is checked~\cite{BartheCsfw04}. For example, to check for information leaks, the low-security variables are initialized to identical values in the two copies of the program, while the high-security variables are unconstrained and can take different values. The safety check ensures that in all executions of the program with the two copies, the values of the low-security variables are identical at the end of the program, i.e., there is no information leak from high-security to low-security variables. The self-composition approach is a general approach for checking hyper-properties, and has been recently applied for checking constant-time implementations of secure 
code~\cite{AlmeidaUsenix16}. 

Although the self-composition reduction is sound and complete, it is challenging to perform safety verification, especially on two copies of a program. An improvement was suggested by Terauchi and Aiken~\cite{TerauchiSas05}, where they combined the self-composition approach with type analysis to make copies of (portions of) the program in a manner that is more friendly for software verifiers. For example, if a loop condition is of type low-security, then the loop bodies are duplicated inside a single loop. This has the effect of keeping corresponding variables in the two copies of the program near each other, which can be useful in deriving invariants to aid safety verification. 

In this paper, we aim to further improve the self-composition approach by making it \emph{lazier}, in contrast to an eager upfront translation into two copies of a program. This lazy duplication is enabled by dynamic taint propagation, which is performed along with an unrolling of the program during bounded model checking (BMC)~\cite{BiereBmc}. Dynamic taint propagation has the benefit of being more precise than static type-based analysis. By combining it with BMC, it is guaranteed to cover all possible (bounded) program executions, unlike other dynamic approaches that cover only the tested executions. This also allows us to leverage existing interpolant-based verification methods for proving correctness over unbounded executions.

We also propose a specialized early termination check for the BMC-based approach. In secure programs, it is often the case that sensitive information is propagated in a very localized context, and conditions exist that squash its propagation any further. An eager self-composition approach does not exploit this, but in effect delays the security property check to take place at the end of the two programs. Indeed, it depends on the software verifier to perform any property-related slicing or other optimizations. We believe that general software verifiers are not equipped to leverage such conditions. We formulate our early termination check in terms of taint queries on live variables during program unrolling. In practice, this check is often successful, leading to much saved effort in comparison to performing a full safety check. 

To summarize, our lazy self-composition approach based on BMC provides the following advantages for security verification:
\begin{itemize}\item It performs \emph{dynamic taint propagation} to determine precise security types while unrolling the program during BMC. This information is more precise than static type-based analysis (which loses precision due to path-insensitive analyses), and covers all possible (bounded) program executions unlike dynamic taint analysis that covers only the tested executions. 
\item It performs \emph{lazy self-composition} by querying security types during program unrolling and using it to avoid or optimize duplicated code. The goals are similar to the Terauchi-Aiken approach, but our dynamic type queries yield more precise information and our duplication optimizations are more effective. (AG: will need to justify this.)
\item It performs an \emph{early termination} check for information flow, whereby it is guaranteed that the security property is correct without any further unrolling.
\end{itemize}
Our security verifier targets rich specifications in terms of low-security and high-security variables/locations, predicates that allow information downgrading in certain contexts, checking for constant-time implementations, etc. It leverages modern SMT-based verification techniques, including (potential) use of interpolants to derive proofs of correctness. We have implemented the security verifier as part of the SeaHorn verification platform~\cite{SeahornCav15}, which represents programs as CHC (Constrained Horn Clause) rules. It has a frontend based on LLVM~\cite{llvm}, and backends to Z3~\cite{z3tool} and mu-Z3~\cite{muz3tool}.


\section{Motivating Examples}
\label{sec:motiv}
% motiv.tex

\noindent\begin{minipage}{\linewidth}
\begin{lstlisting}[
caption={Path-sensitive analysis is more precise.},
label=lst:ex1,
]
sensitive int secret = ??;
int cond = ??;
int i = ??;
	
int answer = secret + 1;
if (cond > 3) answer += 10;
else answer = i + 1;
if (cond <= 3) answer = answer + 10;
else answer = 20; 
\end{lstlisting}
\end{minipage}

Listing~\ref{lst:ex1} shows a small program demonstrating the limitations of static analysis. The program shows a sensitive (confidential) value stored in the variable \code{secret}. The verification goal here is to ensure that \code{secret} does not flow to the untrusted variable \code{answer}. Static analyses, which include analyses based on security types~\cite{TerauchiSas05,MyersPopl99,SabelfeldComm03}, which infer \emph{security types} for each program variable would conclude that the variable \code{answer} is \emph{high security} because each of the two conditional blocks assign sensitive values to \code{answer}. However, a path-sensitive analysis reveals that when $\mcode{cond} > 3$, the sensitive value is erased in line~10. When $\mcode{cond} \leq 3$, a sensitive value is never assigned to \code{answer}.

\begin{lstlisting}[
caption={Dynamic taint analysis in BMC is more precise.},
label=lst:ex2
]
sensitive int secret = ??;
int n = ??;

assume (n > 10);
int answer = secret + 5;
bool flag = 1;
for (int i = 0; i < n; i++) {
  if (flag) answer = 5;
  else answer = answer + 10;
  flag = !flag;
}
\end{lstlisting}

Listing~\ref{lst:ex2} demonstrates the need for a dynamic analysis that goes beyond being path-sensitive. In this example, the confidential  variable \code{secret} is secret again and the verification task is to prove that there is no information flow from \code{secret} to \code{answer}. The Boolean variable \code{flag} controls the value stored in \code{answer}, and in particular if flag is \code{1}, the constant \code{5} is stored into \code{answer}, overwriting any previous information flow. A dynamic analysis which unrolls the loop is required to prove that this program is secure. Dynamic taint analyses~\cite{KangNdss11,SongIss08,CrandallMicro04,SchwartzSp10} can be used to verify such properties, but unfortunately these analyses are restricted to reasoning about one trace and therefore they cannot handle examples where information flow occurs throw control flow, like in Listing~\ref{lst:ex1}.\footnote{This is known as the \emph{implicit flow} problem.}

Listing~\ref{lst:ex3} demonstrates the advantages of early-termination of dynamic analyses. In this example, the information flow from \code{secret} to \code{sum} does not occur after the third loop iteration. An analysis which can infer this fact can terminate early.

\begin{itemize}
\item Example 1 \\
Path-sensitive analysis in MC gives more precise results than static analysis for security types.


\item Example 2 \\
Dynamic taint analysis in BMC gives more precise results than static analysis on loops.

\item Example 3 \\
Advantage of early termination

\end{itemize}

\begin{lstlisting}[
caption={Advantages of Early-Termination.},
label=lst:ex3
]
sensitive int secret = ??;
int n = ??;
assume(n > 10);

int sum = 0;
for (int i = 0; i < n; i++)  {
	if (secret > 3) sum += i + 3;
	else sum -= i - 5;
	if (i >= 2)	sum = i + 10;
}
\end{lstlisting}




\section{Preliminaries}
\label{sec:prelim}
% prelim.tex
% Background and Terminology



\subsection{Safety Verification}
Given a program $A$, we define a \emph{safety verification} problem w.r.t. $A$ as a transition system $M=\langle X, \Init(X),\Tr(X,X'),\Bad(X)\rangle$ where $X$ denotes a set of (uninterpreted) constants, representing program variables; $\Init, \Tr$ and $\Bad$ are (quantifier-free) formulas in $FOL(\cT)$ representing the initial states, transition relation and bad states, respectively. The states of a transition system correspond to structures over a signature $\Sigma = \Sigma_{\cT} \cup X$, 
%We write $\Tr(X,X')$ to denote that $\Tr$ is defined over the signature $\Sigma_{\cT} \cup X \cup X'$, where $X$ is used to represent the pre-state of a transition, and $X' = \{a' \mid a \in X\}$ is used to represent the post-state.

A safety verification problem is to decide whether a transition
system $M$ is SAFE or UNSAFE.

We say that $M$ is UNSAFE iff there exists a
number $N$ such that the following formula is satisfiable:
\begin{equation}
  \Init(X_0) \land \left( \Land_{i=0}^{N-1} \Tr(X_i, X_{i+1}) \right)
  \land \Bad(X_N)
\end{equation}
where $X_i = \{a_i \mid a \in X\}$ is a copy of the constants used to represent the state of the system after the execution of $i$ steps.

When $M$ is UNSAFE and $s_N\in\Bad$ is the reachable state, the path from $s_0\in\Init$ to $s_N$ is called a \emph{counterexample} (CEX).

A transition system $M$ is SAFE iff the transition system has no counterexample, of any length. Equivalently, $M$ is SAFE iff there exists a formula $\Inv$, called a
\emph{safe inductive invariant}, that satisfies:
\begin{align}
  \Init(X) &\to Inv(X) & \Inv(X) \land \Tr(X,X') &\to \Inv(X') & \Inv(X) &\to \neg \Bad(X)
\end{align}

In SAT-based model checking, the verification problem is determined by maintaining an \emph{inductive trace} of formulas $[F_0(X), \ldots, F_N(X)]$ that satisfy:
\begin{align}
  \Init(X) &\to F_0(X)  \\
  \forall 0 \leq i < N \cdot F_i(X) \land \Tr(X,X') &\to F_{i+1}(X') \\
  \forall i \cdot F_i(X) &\to \neg \Bad(X)
\end{align}

A trace $[F_0, \ldots, F_N]$ is \emph{closed} if $\exists 1 \leq i
\leq N \cdot F_i \limp \left( \Lor_{j=0}^{i-1}F_j\right)$. There is an
obvious relationship between existence of closed traces and safety of
a transition system:
\begin{theorem}
  \label{thm:closed-safety}
  A transition system $T$ is SAFE iff it admits a safe closed trace.
\end{theorem}
Thus, safety verification is reduced to searching for a safe closed
trace or finding a CEX.

\yv{Should we note that given a program $A$, a transition system $M$ is created through CHC translation?}

\subsection{Security Verification}

Let $A$ be program.

\subsubsection{Property Specification}
Information flow, downgrading, constant-time and side-channel leaks

\subsubsection{Self-composition}
k-safety, TA approach?



%\section{Lazy Self-Composition}
\label{sec:algo}
% algo.tex

\section{Information Flow Analysis}

Let $A$ be a program over a set of program variables $X$. We assume $\Init(X)$ is a formula describing the initial states and $\Tr(X,X')$ a transition relation. Note that we assume a ``stuttering'' transition relation, namely, $\Tr$ can non-deterministically either move to the next state or stay in the same state. Let us assume that $H \subset X$ is a set of high-security variables and $L := X\backslash H$ is a set of low-security variables. The \emph{information flow} problem checks whether there exists an execution of $A$ such that the value of variables in $H$ propagates to variables in $L$. Intuitively, information flow analysis checks if low-security variables ``leak'' information about high-security variables.

In what follows, we describe two approaches for performing information flow analysis.

\subsection{Taint Analysis}  \label{taint-analysis}

In the setting of taint analysis, we mark high-security variables with a ``taint'' and check if this taint can propagate to low-security variables.
The propagation of taint through program variables of $A$ is determined by assignments in $A$ and the control structure of $A$. In order to perform taint analysis, we formulate it as a safety verification problem. For this purpose, for each program variable $x\in X$, we introduce a new ``taint'' variable $x_t$. Let $X_t := \{x_t | x\in X\}$ be the set of taint variables where $x_t\in X_t$ is of sort Boolean. %We now formulate taint analysis as a safety verification problem. 
Let us define a transition system $M_t := \langle Y, \Init_t,\Tr_t,\Bad_t\rangle$ where $Y := X\cup X_t$ and  
\begin{align}
    \Init_t(Y) & := \Init(X)\land\left(\Land\limits_{x\in H}x_t\right)\land\left(\Land\limits_{x\in L}\neg x_t\right)\\
    \Tr_t(Y,Y') &:= \Tr(X,X')\land\hat{\Tr}(Y,X_t')\\
    \Bad_t(Y) &:= \Lor\limits_{x\in L} x_t
\end{align}

Since taint analysis tracks information flow from high-security to low-security variables, all variables in $H_t$ are initialized to $\true$ while variables in $L_t$ are initialized to $\false$. W.l.o.g., let us denote state update for a program variable $x\in X$ as:

$$x' = cond(X) \; ? \; \varphi_1(X)\; :\; \varphi_2(X)$$

Let $\varphi$ be a formula over $\Sigma$. We capture the taint of $\varphi$ by:

\[ \Theta(\varphi) =
  \begin{cases}
    \false       & \quad \text{if } \varphi\cap X = \emptyset\\
    \Lor\limits_{x\in \varphi} x_t  & \quad \text{otherwise}
  \end{cases}
\]

Based on the above, $\hat{\Tr}(X_t,X_t')$ is defined as follows:
$$ \Land\limits_{x_t\in X_t} x_t' = \Theta(cond)\lor \left( cond\; ? \; \Theta(\varphi_1) \; : \; \Theta(\varphi_2) \right)$$


Intuitively, taint may propagate from $x_1$ to $x_2$ either when $x_1$ is assigned an expression that involves $x_2$ or when an assignment to $x_1$ is controlled by $x_2$. The bad states ($\Bad_t$) are all states where a low-security variable is tainted.

\subsection{Self Composition} \label{self-composition}
In the context of self-composition, information flow is tracked over an execution of two copies of the program, $A$ and $A_s$. Let us assume $X_s := \{x_s | x\in X\}$ is the set of program variables of $A_s$. Similarly, $\Init_s(X_s)$ and $\Tr_s(X_s,X_s')$ are the initial states and transition relation of $A_s$. Note that $\Init_s$ and $\Tr_s$ are computed from $\Init$ and $\Tr$ by means of substitutions. Namely, substituting every occurrence of $x\in X$ or $x'\in X'$ with $x_s\in X_s$ and $x_s'\in X_s'$, respectively. Similarly to taint analysis, we formulate information flow over a self-composed program by a safety verification problem: $M_s := \langle Z, \Init_s,\Tr_s,\Bad_s\rangle$ where $Z := X\cup X_s$ and  
\begin{align}
    \Init_s(Z) & := \Init(X)\land\Init(X_s)\land\left(\Land\limits_{x\in L} x = x_s\right)\\
    \Tr_s(Z,Z') &:= \Tr(X,X')\land\Tr(X_s,X_s')\\
    \Bad_s(Z) &:= \Lor\limits_{x\in L} \neg(x = x_s)
\end{align} 

In order to track information flow, variables in $L_s$  are initialized to be equal to their counterpart in $L$, while variables in $H_s$ remain unconstrained. A leak is captured by the bad states (i.e. $\Bad_s$). More precisely, there exists a leak iff there exists an execution of $M_s$ that ends in a state where a low-security variable $x\in L$ has a different value then its counterpart $x_s\in L_s$.

\yv{Note that we allow for $\Tr$ to be stuttering - meaning, it can either move to the next state, or stay at the same state - this is crucial for correctness of $M_s$}


\section{Lazy Self Composition}

In this section we introduce \Ifc, an algorithm for information flow analysis, that combines both taint analysis and self-composition. 

Recall that taint analysis is imprecise in the sense that may report spurious counterexamples, namely, spurious leaks. In contrast, self composition is precise, but less efficient than taint analysis. The fact that self composition requires a duplication of the program often hinders its performance.

To address both efficiency and precision, \Ifc interchangeably uses taint analysis to guide self composition and vice-versa. As a result, a self composition is applied lazily, when required, making \Ifc more efficient than the naive approach, while at the same time sound and complete.

Before introducing \Ifc, let us introduce the following lemma:

\begin{lemma}
$M_t$ over-approximates $M_s$.
\end{lemma}

\begin{corollary}
If there exists a path in $M_s$ from $\Init_s$ to $\Bad_s$ then there exists a path in $M_t$ from $\Init_t$ to $\Bad_t$.
\end{corollary}

\begin{corollary}
If there exists no path in $M_t$ from $\Init_t$ to $\Bad_t$ then there exists no path in $M_s$ from $\Init_s$ to $\Bad_s$.
\end{corollary}

\Ifc makes use of the fact $M_t$ can be viewed as an abstraction w.r.t. to $M_s$, and follows an abstraction-refinement paradigm for secure information flow analysis. In these setting, $M_t$ is used to find a possible counterexample, i.e. a path that leaks information. Then, $M_s$ is used to check if this counterexample is spurious or real. In case the counterexample is found to be spurious, \Ifc uses the proof that shows why the counterexample is not possible in $M_s$ to refine $M_t$.

A sketch of \Ifc appears in Alg~\ref{alg:ifc}. Recall that we assume that solving a safety verification problem is done by maintaining an inductive trace. We denote the traces for $M_t$ and $M_c$ by $\vG=[G_0,\ldots,G_k]$ and $\vH=[H_0,\ldots,H_k]$, respectively. \Ifc starts by initializing $M_t$, $M_s$  and their respective traces $\vG$ and $\vH$ (lines~\ref{line:init_s}-\ref{line:init_e}). The main loop of \Ifc (lines~\ref{line:main_s}-\ref{line:main_e}) starts by looking for a counterexample over $M_t$ (line~\ref{line:solve_taint}). In case no counterexample is found, \Ifc deduces there are no leaks and returns SAFE.

If a counterexample $\pi$ is found in $M_t$, \Ifc first updates the trace of $M_s$, i.e. $\vH$, by rewriting $\vG$ (line~\ref{line:taint_to_sc}). In order to check if $\pi$ is spurious, \Ifc creates a new safety verification problem $M_c$, a version of $M_s$ constrained by $\pi$ (line~\ref{line:constraint}) and solves it (line~\ref{line:solve_sc}). If $M_c$ has a counterexample, \Ifc returns UNSAFE. Otherwise, $\vG$ is updated by $\vH$ (line~\ref{line:sc_to_taint}) and $M_t$ is refined such that $\pi$ is ruled out (line~\ref{line:refine}).

The above gives a high-level overview of how \Ifc operates. We now go into more detail. 
More specifically, we describe the functions $\texttt{ReWrite}$, $\texttt{Constraint}$ 
and $\texttt{Refine}$. We note that these functions can be designed and implemented in several
ways. In what follows we describe possible choices.

\begin{figure}[!t]
  \begin{algorithm2e}[H]
  \DontPrintSemicolon
  \SetAlgoVlined
  \LinesNumbered
   
  \KwIn{A program $A$ and a set of high-security variables $H$}
  \KwOut{$\texttt{SAFE}$, $\texttt{UNSAFE}$ or $\texttt{UNKNOWN}$.}

  $M_t \gets \texttt{ConstructTaintModel}(A, H)$\\ \label{line:init_s}
  $M_s \gets \texttt{ConstructSCModel}(A, H)$\\
  
  $\vG \gets [G_0=\Init_t]$\\
  $\vH \gets [H_0=\Init_s]$\\ \label{line:init_e}
  \Repeat {$\infty$} {\label{line:main_s}
     $(\vG, R_{taint},\pi)\gets \texttt{MC.Solve}(M_t,\vG)$\label{line:solve_taint}\\
     \uIf{$R_{taint} = \text{SAFE}$} {
        \Return $\texttt{SAFE}$
     }
     \uElse {
        $\vH\gets\texttt{ReWrite}(\vG, \vH)$\label{line:taint_to_sc}\\
        $M_c \gets \texttt{Constraint}(M_s,\pi)$\label{line:constraint}\\
        $(\vH, R_{s},\pi)\gets \texttt{MC.Solve}(M_c,\vH)$\label{line:solve_sc}\\
        \uIf{$R_{s} = \text{UNSAFE}$} {
            \Return $\texttt{UNSAFE}$
        }
     	\uElse {
 	        $\vG\gets\texttt{ReWrite}(\vH, \vG)$\label{line:sc_to_taint}\\
     	    $M_t \gets \texttt{Refine}(M_t, \vG)$\label{line:refine}
     	}
     }
  }\label{line:main_e}
  \Return $\texttt{UNKNOWN}$
  \caption{\texttt{\Ifc(A,H)}}
  \label{alg:ifc}
  \end{algorithm2e}
\end{figure}




\subsection{The Proof-based Abstraction Approach}

Let us assume that when solving $M_t$ a counterexample $\pi$ of length $k$ is found and an inductive trace $\vG$ is computed. Following a proof-based abstraction approach, $\texttt{Constraint}()$ uses the length of $\pi$ to bound the length of possible executions in $M_s$ by $k$. Intuitively, this is similar to bounding the length of the computed inductive trace over $M_s$.

In case $M_c$ has a counterexample, a real leak (of length $k$) is found. Otherwise, since $M_c$ considers all possible executions of $M_s$ of length $k$, \Ifc deduces that there are no counterexamples of length $k$. In particular, the counterexample $\pi$ is ruled out. \Ifc therefore uses this fact to refine $M_t$ and $\vG$.

\subsubsection{Inductive Trace Rewriting}

Consider the set of program variables $X$, taint variables $X_t$, and self compositions variables $X_s$. As noted above, $M_t$ over-approximates $M_s$. Intuitively, it may mark a variable $x$ as tainted (namely, evaluate $x_t$ to $\true$) when $x$ is not really tainted. Equivalently, if a variable $x$ is found 
to be untainted in $M_t$ then it is known to also not leak information in $M_s$. More formally, the following relation holds: $\neg x_t\to  (x = x_s)$.

This gives us a procedure for rewriting a trace over $M_t$ to a trace over $M_s$. Let $\vG=[G_0,\ldots,G_k]$
be an inductive trace over $M_t$. Considering the definition of $M_t$, $\vG$ can be written as:
$$ G_i(Y) := \bar{G}_i(X)\land \bar{G}^t_i(X_t)\land\psi(X,X_t) $$

Following the definition of an inductive trace, the following holds:
$$\bar{G}_i(X)  \land \Tr(X,X') \to \bar{G}_{i+1}(X')$$


Let $\vH = [H_0,\ldots,H_k]$ be a trace w.r.t. $M_s$. $\vH$ can be strengthened by $\vG$ as follows: 
\begin{align}
    H_0 &:= \Init_s \\
    H_i(Z) &:= H_i(Z)\land \bar{G}_i(X)\land\bar{G}_i(X_s)\land\left(\Land\{x = x_s | G_i(Y)\to\neg x_t \}\right)
\end{align}

\begin{lemma}
Let $\vG$ be an inductive trace w.r.t. $M_t$ and $\vH$ an inductive trace w.r.t. $M_s$. Then, the updated $\vH$ defined above is an inductive trace w.r.t. $M_s$.
\end{lemma}

A similar strengthening can be defined when updating a trace $\vG$ w.r.t. $M_t$ by a trace $\vH$ w.r.t. $M_s$. In this case, we use the following relation: $\neg(x = x_s)\to x_t$. Let $\vH=[H_0(Z),\ldots,H_k(Z)]$ be the inductive trace w.r.t. $M_s$. $\vH$ can be written as
$$H_i(Z) := \bar{H}_i(X)\land \bar{H}_i^s(X_s)\land \phi(X,X_s)$$

Due to the definition of $M_s$ and an inductive trace, the following holds:
    $$\bar{H}_i(X)\land\Tr(X,X')\to\bar{H}_i(X')$$  $$\bar{H}^s_i(X_s)\land\Tr(X_s,X_s')\to\bar{H}^s_i(X_s')$$
    
We can therefore strengthen a trace $\vG = [G_0,\ldots,G_k]$ w.r.t. $M_t$ by:
\begin{align}
    G_0 &:= \Init_s \\
    G_i(Y) &:= G_i(Y)\land\bar{H}_i(X)\land\bar{H}^s_i(X)\land\left(\Land\{ x_t | H_i(Z)\to \neg (x = x_s) \}\right)
\end{align}

Note that the above strengthening of $\vG$ by $\vH$, and vice-versa, is based on the fact that $M_t$ over-approximates $M_s$ w.r.t. tainted variables. It therefore cannot add more ``precision'' to $\vG$.

\subsubsection{Refinement}

Recall that in the current scenario, a counterexample was found in $M_t$, and was shown to be 
spurious in $M_s$. This fact can be used to refine both $M_t$ and $\vG$.

As a first step, we observe that if $x = x_s$ in $M_s$, then $\neg x_t$ should hold in $M_t$.
However, since $M_t$ is an over-approximation it may allow $x$ to be tainted, namely, allow $x_t$
to be evaluated to $\true$.

Based on the above, we reformulate the strengthening procedure for $\vG$. 
Let $\vH = [H_0,\ldots,H_k]$ be a trace w.r.t. $M_s$ and $\vG = [G_0,\ldots, G_k]$
be a trace w.r.t. $M_t$, then the strengthening of $\vG$ is defined by:

\begin{align}
    G_0    := & \Init_s \\
    \begin{split}
    G_i(Y) := & G_i(Y)\land\bar{H}_i(X)\land\bar{H}^s_i(X)\land\left(\Land\{ x_t | H_i(Z)\to \neg (x = x_s) \}\right)\land\\
    & \left(\Land\{ \neg x_t | H_i(Z)\to (x = x_s) \}\right)
    \end{split}
\end{align}

The above gives us a procedure for strengthening $\vG$ by using $\vH$. Note that $\vG$ is not necessarily an inductive trace w.r.t. $M_t$ since it may be the case that $G_i\land\Tr_t\to G_{i+1}'$ does not hold. This is due to $M_t$ being an over-approximation. Therefore, after $\vG$ is strengthened, $M_t$ must be refined such that $\vG$ is an inductive trace w.r.t. $M_t$.

Let us assume that $G_i\land\Tr_t\to G_{i+1}'$ does not holds. By that, $G_i\land\Tr_t\land\neg G_{i+1}'$ is satisfiable. Considering the way $\vG$ is strengthened,
three exists $x\in X$ such that $\neg x_t\in G_{i+1}$ and $G_i\land\Tr_t\land x_t'$ is satisfiable. The refinement step is defined by:

$$ x_t' = G_i\;? \; \false \; : \left( \Theta(cond)\lor \left( cond\; ? \; \Theta(\varphi_1) \; : \; \Theta(\varphi_2) \right)\right)$$


\subsection{BMC based Approach}
In this section we introduce a different approach of solving the information flow security problem that is based on Bounded Model Checking (BMC). This approach is described in Alg~\ref{alg:ifc-bmc}.
To perform the \Bmc procedure users should provide an extra parameter $BND$ which will limit the maximum number of loop unrolls performed on the program $A$, avoiding a possibly endless search. The unroll bound used in \Bmc is incremented repeatedly until reaching the maximum. 

During each iteration of the algorithm (line~\ref{line:bmc-loop}), loops in the program $A$ are unrolled once more (line~\ref{line:bmc-unroll}) to produce a loop-free program, encoded as a transition system $M(i)$. A new transition system $M_t(i)$ is created (line~\ref{line:bmc-taint}) following the method described in section~\ref{taint-analysis}, in order to capture all the taint propagation that may happen in the unrolled program $M(i)$. Self composition is applied lazily on $M(i)$, based on the results of taint analysis through evaluating the states in $M_t(i)$. The procedure is described in section~\ref{self-composition} but in this BMC based implementation (line~\ref{line:lazysc}) we try our best effort to reduce the complexity of the self-composed program. In detail, for each variable $x$ appearing in $M(i)$, we denote the state update to be $x := \varphi$ and $x_t$ to be the corresponding variable in $M_t(i)$ which encodes whether $x$ could be tainted or not. If $x_t$ evaluates to $False$ we conclude that high security variables do not affect the value of $x$, therefore its duplicate variable $x'$ in the self-composed program $M_s(i)$ should hold the same value, eliminating the need to duplicate the computation that will produce $x'$. Otherwise if $x_t$ is $True$ or $UNKNOWN$, we will have to duplicate $\varphi$ and replace the occurrences of all program variables with their corresponding duplicates. Therefore we denote the state update for $x'$ in the self-composed program $M_s(i)$ to be $x' := \varphi'$:
\[ \varphi' =
  \begin{cases}
    x       & \quad \text{if } x_t \text{ evaluates to } False\\
    duplicate(\varphi)  & \quad \text{otherwise}
  \end{cases}
\]


The self-composed program created by \texttt{LazySC} (line~\ref{line:lazysc}) is then verified by a model checker (in our case, the Z3 SMT solver). A counter example produced by the solver indicates a leak in the original program $A$. We have also implemented  liveness checks on taint, that can allow early termination of \Bmc if all the taint on live variables has been squashed during execution of the program. In either case \Bmc will stop unrolling the program any further. If no conclusive result can be acquired before reaching the unroll bound, \Bmc will return $UNKNOWN$.

\begin{figure}[!t]
  \begin{algorithm2e}[H]
  \DontPrintSemicolon
  \SetAlgoVlined
  \LinesNumbered
   
  \KwIn{A program $A$ and a set of high-security variables $H$, max unroll bound $BND$}
  \KwOut{$\texttt{SAFE}$, $\texttt{UNSAFE}$ or $\texttt{UNKNOWN}$.}

  $i \gets 0$\\
  \Repeat {$i = BND$} {\label{line:bmc-loop}
  	 $M(i) \gets \texttt{LoopUnroll}(A, i)$ {\label{line:bmc-unroll}}\\
  	 $M_t(i) \gets \texttt{EncodeTaint}(M(i))$ {\label{line:bmc-taint}}\\
  	 $M_s(i) \gets \texttt{LazySC}(M_t(i), M(i))$ {\label{line:lazysc}}\\
  	 $result \gets \texttt{Solve}(M_s(i))$\\
     \uIf{$result = \text{counter-example}$} {
        \Return $\texttt{UNSAFE}$
     }
     $live\_taint \gets \texttt{CheckLiveTaint}(M_t(i))$\\
     \uIf{$live\_taint = \text{false}$} {
      	\Return $\texttt{SAFE}$  
     }
     $i \gets i+1$\\
  }
  \Return $\texttt{UNKNOWN}$
  \caption{\texttt{\Bmc(A,H,BND)}}
  \label{alg:ifc-bmc}
  \end{algorithm2e}
\end{figure}



\begin{comment}
\subsection{Dynamic Taint Propagation}

\subsection{Lazy Duplication}

\subsection{Early Termination}

\subsection{Interpolant-based Verification}
\end{comment}

\section{Implementation and Experiments}
\label{sec:exp}
% exp.tex

\subsection{Implementation}
\subsection{Experiments: \Bmc}

\begin{table}
    \centering
    \begin{tabular}{cccccc}
        \toprule
        \textbf{Benchmark} & \textbf{Result} & \textbf{Time (baseline) } & \textbf{Time (\Bmc)} & \textbf{Time (\Bmc taint checks)} & Description \\
        \midrule
        fibonacci         &   SAFE   & 0.55 & 0.1   & 0.07   & N-th Fibonacci number\\ \midrule
        list\_fakemalloc4  &   SAFE   & 2.9  & 0.15  & 0.007  & linked list operations, maximum 4 nodes\\ \midrule
        list\_fakemalloc8  &   SAFE   & 3.1  & 0.6   & 0.02   & same as above, maximum 8 nodes\\ \midrule
        list\_fakemalloc16 &   SAFE   & 3.2  & 1.83  & 0.08   & same as above, maximum 16 nodes\\ \midrule
        modexp\_safe       &   SAFE   & TO   & 0.05  & 0.01   & modular exponentiation, no leak\\ \midrule
        modexp\_unsafe     & UNSAFE   & TO   & 1.63  & 1.5    & modular exponentiation, with leaks\\ \midrule
        pwdcheck\_safe16   &   SAFE   & TO   & 0.05  & 0.01   & string comparison, no leak, 16 characters\\ \midrule
        pwdcheck\_unsafe16 & UNSAFE   & TO   & 1.63  & 1.5   & string comparison, with leaks, 16 characters\\
        \bottomrule
    \end{tabular}   
    \caption{\Bmc experiments}
    \label{tab:exp}
\end{table}

We tested our \Bmc implementation on several micro-benchmarks in addition to benchmarks inspired by real-world programs. The performance of \Bmc is compared with the baseline of eager two-copy translation as shown in Tab~\ref{tab:exp}.

\subsection{Experiments: Micro-Benchmarks}
show scalability for increasing parameter values, compare with baseline (eager 2-copy translation)

\begin{table}
    \centering
    \begin{tabular}{ccccc}
        \toprule
        \textbf{Benchmark} & \textbf{Parameter} & \textbf{Baseline (SC)} & \textbf{Taint Analysis} & \textbf{Taint Analysis + SC} \\
        \midrule
        \multirow{4}{*}{Password Check/Safe} 
        &   4   &          8.8 &        0.2 &    N/A \\
        &   8   & TO          & 0.2     & N/A \\
        &   16  & TO          & 0.2     & N/A \\
        &   32  & TO          & 0.2     & N/A \\
        \midrule
        \multirow{2}{*}{Modular Addition} 
        & 2048b & 1.0 & 0.2 & N/A \\
        & 4096b & TA  & 0.3 & N/A \\
        \bottomrule
    \end{tabular}   
    \caption{Results}
    \label{tab:results}
\end{table}
\begin{center}
\end{center}
\subsection{Experiments: Case-studies}
Pramod: please select which examples are inspired by real case studies \\
We should try to find a few more (2-3?) examples from recent papers: Almeida Usenix16, Dillig PLDI16 \\








\section{Related Work}
\label{sec:related}
% related.tex

\begin{itemize}

\item Secure information flow \\
confidentiality, integrity, non-interference

\item Language-based security, static type analysis 

\item Dynamic taint analysis\\
dynamic execution, symbolic execution 

\item Self-composition approaches\\
Barthe et al., hyper-LTL checker, recent papers

\end{itemize}






\section{Conclusions and Future Directions}
\label{sec:concl}
% concl.tex









\bibliographystyle{splncs03}  

\bibliography{b1}

\end{document}
